\usepackage[latin1]{inputenc} % acentuacao
\usepackage{graphicx} % para inserir figuras

\hypersetup{
    hidelinks, % Remove coloração e caixas
    unicode=true,   %Permite acentuação no bookmark
    linktoc=all %Habilita link no nome e página do sumário
}

\chapter{Agglomerative Clustering}
	Benchmarking é uma prática que consiste em criar programas, operações ou casos de teste para testar o desempenho de um objeto, seja ele software ou hardware (SAAVE-DRA;SMITH, 1996), essa técnica é amplamente abordade em diversas áreas da computação. Nesse trabalho será abordado o desenvolvimento de um benchmark do Agglomerative Clustering.
	O Agglomerative Clustering é um algoritmo bottom-up, que recebe como entrada um data-set de pontos em um espaço n-dimenional e uma função que mede a similaridade entre os itens desse data-set, normalmente essa função representa uma métrica de distância entre os pontos no espaço, onde os pontos mais semelhantes e encontram mais próximos.
	O algoritmo trabalha agrupando os pontos mais próximos e, com isso, gera uma Binary Cluster Tree (BCT), que é o resultado do agrupamento dos dados.
